\documentclass{beamer}
\usepackage{amsmath}
\usepackage{amssymb}
\usepackage{mathrsfs}
\usepackage{amsthm}
\usepackage[utf8]{inputenc}
\usepackage{thmtools,thm-restate}
\usepackage{bm}
\usepackage[italian]{babel}
\usepackage{graphicx}
\usepackage{textpos}
\usepackage{bbm}


\usetheme{Madrid}
\setbeamertemplate{blocks}[rounded][shadow=true]
\setbeamertemplate{footline}
{
  \leavevmode%
  \hbox{%
  \begin{beamercolorbox}[wd=.4\paperwidth,ht=2.25ex,dp=1ex,center]{author in head/foot}%
    \usebeamerfont{author in head/foot}\insertshortauthor
  \end{beamercolorbox}%
  \begin{beamercolorbox}[wd=.6\paperwidth,ht=2.25ex,dp=1ex,center]{title in head/foot}%
    \usebeamerfont{title in head/foot}\insertshorttitle\hspace*{3em}
    \insertframenumber{} / \inserttotalframenumber\hspace*{1ex}
  \end{beamercolorbox}}%
  \vskip0pt%
}

\newcommand{\bbR}{\mathbb{R}}
\newcommand{\bbZ}{\mathbb{Z}}
\newcommand{\de}{\partial}
\newcommand*\diff{\mathop{}\!\mathrm{d}}
\newcommand{\OO}{\mathcal{O}}
\newcommand{\AAA}{\mathcal{A}}
\newcommand{\F}{\mathcal{F}}
\newcommand{\Ft}{\widetilde{\F}}
\newcommand{\R}{\mathcal{R}}
\newcommand{\Id}{\mathrm{Id}}
\newcommand{\1}{\mathbbm{1}}

 
%Information to be included in the title page:
\title{Sample title}
\author[Marco Miani]{Candidato: Marco Miani\\ Relatore: Prof. Luigi Carlo Berselli}
\institute{Università di Pisa}
\date{26 ottobre 2018}
 \title[Dal problema di Calder\'on al mantello di Harry Potter] %optional
{Dal problema di Calder\'on al mantello di Harry Potter}

\begin{document}
 
\begin{frame}[plain]
	\centering\includegraphics[scale=0.03]{logounipi.png}
	\maketitle
\end{frame}


\begin{frame}{Premessa}
Le equazioni di Maxwell determinano l'intera elettrodinamica.
\[
\left\{\begin{array}{l}
         \nabla\cdot D = 4\pi\rho \\
         \nabla \times E = - \frac{1}{c}\frac{\de B}{\de t} \\
         \nabla\cdot B = 0 \\
         \nabla \times H = \frac{4\pi}{c}J + \frac{1}{c}\frac{\de D}{\de t} 
    \end{array}\right.
\]
e valgono le relazioni
\[
D=\epsilon E
\qquad
\mu H=B
\qquad
J=\sigma E
\]
Quindi $\sigma,\epsilon,\mu:\bbR^3\rightarrow\mathfrak{M}_{3\times 3}(\bbR)$ sono gli unici parametri rilevanti.
\end{frame}


\begin{frame}{Il problema dell'elettrostatica}
Sia $\Omega\subset\bbR ^n$ un compatto, la nostra regione di misura.

\pause
Sia $\sigma: \Omega \rightarrow\mathfrak{M}_{n\times n}(\bbR)$ una condicibilità elettrica\pause, cioè t.c.
\begin{itemize}
\item $\sigma(x)^T=\sigma(x) \qquad \forall x\in\Omega;$
\pause
\item $\langle\sigma(x)\xi,\xi\rangle \geq 0 \qquad \forall\xi\in\bbR^n ,\, \forall x\in\Omega.$
\end{itemize}

\vspace{10pt}
\pause
In assenza di cariche in $\Omega$ il potenziale elettrico $u$ è soluzione di
\begin{equation*}
\left\{ \begin{array}{ll}
         \nabla\cdot(\sigma\nabla u)=0 & \text{in } \Omega \\
         u|_{\de\Omega}=f
    \end{array}
\right.
\end{equation*}
quando imponiamo condizioni di Dirichlet $f\in H^{1/2}(\de\Omega)$.
\end{frame}


\begin{frame}{Il problema inverso di Calder\'on}
\begin{block}{Definizione}
Assegnata una conducibilità $\sigma$ definiamo la mappa Dirichlet-To-Neumann
\begin{equation*}
\Lambda_\sigma :\quad u|_{\de\Omega} \longmapsto (\sigma\nabla u)\cdot \textbf{n} |_{\de\Omega}.
\end{equation*}
\end{block}
\pause
\vspace{20pt}
Il problema di Calder\'on consiste nel chiedersi se la mappa
\[ \mathcal{C} : \quad \Lambda_\sigma \longmapsto \sigma \]
sia ben definita.

\pause
La risposta, come vedremo, è in generale negativa.
\end{frame}


\begin{frame}{Tomografia ad Impedenza Elettrica}
\begin{block}{Caso isotropo}
%Se si assume che la conducibilità sia isotropa, finita e strettamente positiva 
Se si hanno $\sigma_i:\Omega\rightarrow(0,+\infty)$ per $i=1,2$, vale
\[
\Lambda_{\sigma_1} = \Lambda_{\sigma_2} \quad\Longrightarrow\quad \sigma_1=\sigma_2
\]
La mappa $\mathcal{C}$ è quindi ben definita in questo caso.
\end{block}
Questo costituisce il fondamento della Tomografia ad Impedenza Elettrica.

\pause
\centering
%\includegraphics[width=0.7\linewidth]{EIT.png}
\begin{picture}(80,105)%
    \put(-100,0){\includegraphics[width=0.3\linewidth]{EIT1.png}}%
    \put(43,65){\color[rgb]{0,0,0}\makebox(0,0)[lb]{\smash{$\mathcal{C}$}}}%
    \put(38,55){\color[rgb]{0,0,0}\makebox(0,0)[lb]{\smash{$\longrightarrow$}}}%
    \put(80,0){\includegraphics[width=0.3\linewidth]{EIT2.png}}%
\end{picture}
\end{frame}


\begin{frame}{Tomografia ad Impedenza Elettrica - Esempi espliciti}
Per $\alpha\geq0$ e $0<\rho<1$ consideriamo la conducibilità nella palla $B_1\subset\bbR^2$
\begin{equation*}
\sigma_{\alpha,\rho}(x) = 
	\left\{\begin{array}{lll}
         \alpha & & \text{se } x\in B_\rho \\
         1 & & \text{se } x\in B_1 \setminus B_\rho
    \end{array}\right..
\end{equation*}
\vspace{-15pt}
\pause
\begin{block}{Caso $\alpha=0$}
\[
\Lambda_{\sigma_{0,\rho}}(f) = 
\sum_{k\in\mathbb{Z}\setminus\{0\}} |k|\frac{1-\rho^{2|k|}}{1+\rho^{2|k|}} \hat{f}_k e^{ik\theta} 
\]
\end{block}
\pause
\begin{block}{Caso $\alpha=\infty$}
\[
\Lambda_{\sigma_{\infty,\rho}}(f) = 
\sum_{k\in\mathbb{Z}\setminus\{0\}} |k|\frac{1+\rho^{2|k|}}{1-\rho^{2|k|}} \hat{f}_k e^{ik\theta}
\]
\end{block}
\pause
dove $\hat{f}_k = \frac{1}{2\pi} \int_0^{2\pi} f(\theta)e^{-ik\theta} \diff\theta.$
\end{frame}


\begin{frame}{Invarianza per cambio di variabili}
\pause
\begin{block}{Teorema (del cambio di variabili)}
Sia $F:\overline{\Omega}\rightarrow\overline{\Omega}$ un diffeomorfismo $C^1$ tale che $F|_{\de\Omega}=Id$.\pause

Sia $\tilde{\sigma}$ il push-forward di $\sigma$ tramite $F$
\[ \tilde{\sigma}(y):=F_*\sigma(y)=\frac{(DF(x))^T\sigma(x)(DF(x))}{\det(DF(x))}\Bigg|_{x=F^{-1}(y)}.
\]\pause
Allora vale
%\begin{Large}
\[ \Lambda_{\tilde{\sigma}} = \Lambda_\sigma ,\]
%\end{Large}
cioè le due conducibilità sono assolutamente indistinguibili tramite sole misurazioni esterne.
\end{block}
\end{frame}


\begin{frame}{Invarianza per cambio di variabili - Idea dimostrazione}
\pause
\begin{block}{Definizioni}
Sia\onslide<4,5>{no} $\mathcal{Q}_\sigma: H^{1/2}(\de\Omega) \rightarrow \bbR$ \onslide<4,5>{\quad e $\mathcal{P}_\sigma(u): H^1(\Omega)\rightarrow\bbR$ date da}
\vspace{-5pt}
\[
\mathcal{Q}_\sigma(f) := \int_{\de\Omega} f\,\Lambda_\sigma(f) \diff S
\qquad
\onslide<4,5>{\mathcal{P}_\sigma(u) :=\int_\Omega \langle \sigma \nabla u, \nabla u \rangle \diff x}
\]
\end{block}
\pause
\begin{block}{Lemma}
Sia $\Omega\subset\bbR^n$ un compatto e siano $\sigma$ e $\eta$ due conducibilità su $\Omega$. Vale
\vspace{-10pt}
\[
\mathcal{Q}_{\sigma} = \mathcal{Q}_{\eta}  \quad\Longleftrightarrow\quad  \Lambda_\sigma = \Lambda_\eta.
\]
\end{block}
\pause
\pause
\begin{block}{Lemma}
Data $f\in H^{1/2}(\de\Omega)$ le seguenti condizioni su $u\in H^1(\Omega)$ sono equivalenti:
\vspace{-5pt}
\[
\left\{\begin{array}{lll}
         \nabla\cdot(\sigma\nabla u) =0& & \text{in }\Omega \\
         u|_{\de\Omega}=f & & 
    \end{array}\right.,
\qquad\qquad
\mathcal{P}_\sigma(u)=\min_{v\in H^1(\Omega), v|_{\de\Omega}=f} \mathcal{P}_\sigma(v).
\]
\end{block}
\end{frame}


\begin{frame}{Costruzione Cloaking non singolare}
Consideriamo $\Omega=B_2$.

\pause
\vspace{10pt}
Prendiamo un qualsiasi oggetto contenuto nella palla di raggio $1$ e sia $\OO: B_1\rightarrow \mathfrak{M}_{n\times n}(\bbR)$ la sua conducibilità in ciascun punto.

\vspace{10pt}
\onslide<3>{Per nascondere l'oggetto supponiamo di avvolgerlo con un guscio sferico $B_2\setminus B_1$  con una conducibilità ancora da stabilire.}
\vspace{-40pt}
\begin{columns}
\begin{column}{0.4 \textwidth}
\[ 
\qquad\sigma_\OO(x) = 
	\left\{\begin{array}{lll}
         \OO(x) & & \text{se } x\in B_1 \\
         \onslide<3>{\AAA(x) & & \text{se } x\in B_2\setminus B_1}
    \end{array}\right.
\]
\end{column}
\begin{column}{0.6 \textwidth}
\begin{figure}[H]
%\def\svgwidth{0.8\linewidth}
\begin{picture}(200,200)%
    \put(70,45){\includegraphics[width=0.6\linewidth]{cerchi.jpg}}%
    \put(130,120){\color[rgb]{0,0,0}\makebox(0,0)[lb]{\smash{$\OO(x)$}}}%
    \onslide<3>{\put(140,150){\color[rgb]{0,0,0}\makebox(0,0)[lb]{\smash{$\AAA(x)$}}}}%
\end{picture}
\end{figure}
\end{column}
\end{columns}
\end{frame}


\begin{frame}{Costruzione Cloaking non singolare}
Fissiamo un parametro $0<\rho<1$ e consideriamo la seguente funzione radiale $\F_\rho:\overline{B_2}\rightarrow \overline{B_2}$
\begin{equation*}
\F_\rho(x) = 
	\left\{\begin{array}{lll}
         \frac{x}{\rho} & & \text{se } x\in B_\rho \\
         \left(1+\frac{|x|-\rho}{2-\rho}\right)\frac{x}{|x|} & & \text{se } x\in \overline{B_2}\setminus B_\rho
    \end{array}\right..
\end{equation*}
\pause
\begin{figure}[H]
%\def\svgwidth{0.8\linewidth}
\begin{picture}(100,100)%
    \put(-100,0){\includegraphics[width=0.3\linewidth]{AABcerchiPRE.jpg}}%
    \put(43,75){\color[rgb]{0,0,0}\makebox(0,0)[lb]{\smash{$\Ft_\rho$}}}%
    \put(40,60){\color[rgb]{0,0,0}\makebox(0,0)[lb]{\smash{$\longrightarrow$}}}%
    \put(100,0){\includegraphics[width=0.3\linewidth]{AABcerchiPOST.jpg}}%
\end{picture}
\end{figure}
\end{frame}


\begin{frame}{Costruzione Cloaking non singolare}
Sia $\1:\overline{B_2}\setminus B_\rho \rightarrow \bbR$ la conducibilità isotropa che vale identicamente $1$.
\pause
\begin{block}{Situazione effettiva}
\begin{equation*}
\sigma_\OO(x) = 
	\left\{\begin{array}{lll}
         \OO(x) & & \text{se } x\in B_1 \\ \pause
         \left(\Ft_\rho|_{B_2\setminus B_\rho}\right)_*\!\1\,(x) & & \text{se } x\in \overline{B_2}\setminus B_1
    \end{array}\right.
\end{equation*}
\end{block}
\pause
\begin{block}{Situazione apparente}
\begin{equation*}
(\Ft_\rho^{-1})_*\sigma_\OO(x) = 
	\left\{\begin{array}{lll}
         \left(\Ft_\rho^{-1}|_{B_1}\right)_*\!\OO\,(x) & & \text{se } x\in B_\rho \\
         \1(x) & & \text{se } x\in \overline{B_2}\setminus B_\rho
    \end{array}\right.
\end{equation*}
\end{block}
\pause
Per il Teorema sull'invarianza per cambio di variabili concludiamo che 
\[
\Lambda_{\sigma_\OO} = \Lambda_{(\Ft_\rho^{-1})_*\sigma_\OO}.
\]
\end{frame}


\begin{frame}{Stima bontà della soluzione}
\begin{block}{Definizione}
Sia $\Vert \,.\, \Vert$ la norma sugli operatori $\Lambda: H^{1/2}(\de\Omega) \rightarrow H^{-1/2}(\de\Omega)$ data da
\[
\Vert\Lambda_\sigma\Vert := \sup\left\{|\mathcal{Q}_\sigma(f)|   :\,\Vert f\Vert_{H^{1/2}(\de\Omega)}\leq1\right\}.
\]
\end{block}

\pause
\begin{block}{Lemma}
Date $\sigma$ e $\eta$ conducibilità, cioè funzioni su $\Omega$ a valori matriciali simmetrici
\[ 
	\left. \begin{array}{c}
         \langle\sigma(x)\xi,\xi\rangle \;\le\; \langle\eta(x)\xi,\xi\rangle \\
         \forall x\in\Omega, \forall\xi\in\mathbb{R}^n
    \end{array}
\right|\implies
	\begin{array}{c}
         \mathcal{Q}_\sigma(f) \le \mathcal{Q}_\eta(f) \qquad \forall f\in H^{1/2}(\de\Omega)
    \end{array}.
\]
\end{block}
\pause
\[
\langle\sigma_{0,\rho}(x)\xi,\xi\rangle 
\,\le\, 
\langle(\Ft_\rho^{-1})_*\sigma_\OO(x)\xi,\xi\rangle 
\,\le\, 
\langle\sigma_{\infty,\rho}(x)\xi,\xi\rangle
\qquad
         \forall x\in\Omega, \forall\xi\in\mathbb{R}^n.
\]
\end{frame}


\begin{frame}{Stima bontà della soluzione}
In dimensione 2 si può calcolare esplicitamente la norma di $\Lambda_{\sigma_{0,\rho}}\!\!- \Lambda_\1$ e $\Lambda_{\sigma_{\infty,\rho}}\!\!- \Lambda_\1$, ma in generale vale
\pause
\begin{block}{Teorema}
Sia $B_1\subset\bbR^n$ la palla di raggio $1$ con conducibilità $\sigma_{\alpha,\rho}$ definita prima, per ogni $n$ esiste una costante $C>0$ tale che
\[
\Vert\Lambda_{\sigma_{0,\rho}} - \Lambda_\1\Vert \le C\rho^n\Vert\Lambda_\1\Vert
\]
\[
\Vert\Lambda_{\sigma_{\infty,\rho}} - \Lambda_\1\Vert \le C\rho^n\Vert\Lambda_\1\Vert.
\]
\end{block}
\pause
e grazie all'ordinamento otteniamo che per ogni conducibilità $\OO(x)$
\[
\Vert\Lambda_{\sigma_\OO} - \Lambda_\1\Vert
\le
C\rho^n\Vert\Lambda_\1\Vert
\]
\end{frame}


\begin{frame}{Studio degli autovalori}
\begin{columns}
\begin{column}{.05 \textwidth}
\end{column}
\begin{column}{.4 \textwidth}
Consideriamo il caso in dimensione 2 e guardiamo $\F_\rho$ in coordinate polari
\[
\F_\rho\big((r,\theta)\big) = (\R_\rho(r),\theta).
\]
\end{column}
\pause
\begin{column}{.6 \textwidth}
\begin{figure}[H]
\def\svgwidth{\linewidth}
\input{drawing.pdf_tex}
\end{figure}
\end{column}
\end{columns}

%\vspace{-30pt}
\pause
\hrule
\vspace{10pt}
Il push-forward della conducibilità $\1$ è
\[
\F_*\1\big((r,\theta)\big) = 
\frac{(D\F)^T(D\F)}{\det(D\F)} \circ \F^{-1} (r,\theta)
=
	\left(
		\begin{array}{cc}
         \R'\big(\R^{-1}(r)\big)  &  0 \\
         0  &  \frac{1}{\R'\big(\R^{-1}(r)\big)}
    	\end{array}
    \right).
\]
\end{frame}


\begin{frame}{Cloaking Singolare}
\pause
Facciamo il limite puntuale di $\F_\rho$ per $\rho\rightarrow0$, questo limite esiste ovunque ad esclusione del punto $0$. 
\pause
Otteniamo $\F:\overline{B_2}\setminus\{0\} \rightarrow \overline{B_2}\setminus \overline{B_1}$ data da
\[
\F(x):
=\lim_{\rho\rightarrow0}\F_\rho(x)
= \left(1 + \frac{|x|}{2}\right) \frac{x}{|x|}.
\]
\pause
Questo non è più nemmeno un diffeomorfismo di $\overline{B_2}$ in se, e quindi non possiamo più utilizzare il Teorema sul cambio di variabili.
\end{frame}


\begin{frame}{Condizioni necessarie}
\pause
\begin{block}{Teorema (Astali, Päivärinta)}
Sia $\Omega\subset\bbR^2$ un dominio con frontiera $C^3$, siano $\sigma, \eta \in C^3(\Omega)$ due conducibilità anisotrope, finite e strettamente positive.\\ \pause
Se vale
\[
\Lambda_{\sigma} = \Lambda_{\eta}
\]
allora esiste un diffeomorfismo $\F:\Omega\rightarrow\Omega$ con $\F|_{\de\Omega}=Id$ tale che
\[
\sigma=F_*\eta.
\]
\end{block}
\pause
Questo teorema è stato dimostrato anche in dimensione $n\geq3$ da Lee e Uhlmann con l'ipotesi aggiuntiva che la frontiera e le conducibilità siano analitiche.
\end{frame}


\begin{frame}{Costruzione Cloaking singolare}
\begin{block}{Teorema}
Sia $\F: \overline{B_2}\setminus\{0\} \rightarrow  \overline{B_2}\setminus \overline{B_1}$ radiale un diffeomorfismo $C^\infty$ tale che $\F|_{\de\Omega}=Id$, tale che $\lim_{|y|\rightarrow1} (\F_*\mathbbm{1}(y))\textbf{n} = 0$ e tale che $D\F(x)\geq c_0I$.\pause

Sia $\sigma_\OO$ definita da
\[
\sigma_\OO(x) = 
	\left\{\begin{array}{lll}
         \OO(x) & & \text{se } x\in B_1 \\
         \F_*\1(x) & & \text{se } x\in B_2\setminus B_1
    \end{array}\right.,
\]
con $\OO(x)$ arbitrario ma finito e strettamente positivo.\pause

Allora vale
%\begin{Large}
\[ \Lambda_{\sigma_\OO} = \Lambda_\1 ,\]
%\end{Large}
cioè le due conducibilità sono assolutamente indistinguibili tramite sole misurazioni esterne.
\end{block}
\end{frame}


\begin{frame}{Costruzione Cloaking singolare - Idea dimostrazione}
\pause
Il sistema
\begin{equation*}
\left\{ \begin{array}{ll}
         \nabla\cdot(\sigma_\OO\nabla v)=0 & \text{in } \Omega \\
         v|_{\de\Omega}=f
    \end{array}
\right.
\end{equation*}
\pause
ammette come unica soluzione
\begin{equation*}
v(y) = 
	\left\{\begin{array}{lll}
         w(\F^{-1}(y)) & & \text{per } y\in B_2\setminus B_1 \\
         w(0) & & \text{per } y\in B_1
    \end{array}\right.,
\end{equation*}
con $w$ funzione armonica con le stesse condizioni di Dirichlet, cioè
\[
\nabla\cdot\nabla w=0 \text{ in }B_2 
\qquad
\qquad
\qquad
w=f \text{ in }\de B_2.
\]
\end{frame}


\begin{frame}{Generalizzazione a forme arbitrarie}
\begin{block}{Teorema}
Sia $\Omega\subset\bbR^n$ un compatto connesso. Sia $\mathcal{G}:B_2\rightarrow\Omega$ una mappa Lipschitziana con inversa Lipschitziana e sia $D:=\mathcal{G}(B_1)$. \pause Allora $\mathcal{H}:=\mathcal{G}\circ\F\circ\mathcal{G}^{-1}: \Omega\rightarrow\Omega$ agisce come l'identità su $\de\Omega$ mentre espande il punto $z_0=\mathcal{G}(0)$ a tutto $D$. Sia $\sigma_\OO:\Omega\rightarrow\mathfrak{M}_{n\times n}(\bbR)$ definita da
\[
\sigma_\OO(x) = 
	\left\{\begin{array}{lll}
         \OO(x) & & \text{se } x\in D \\
         \mathcal{H}_*\1(x) & & \text{se } x\in \Omega\setminus D
    \end{array}\right.,
\]
con $\OO(x)$ arbitrario ma finito e strettamente positivo. \pause

Allora la mappa Dirichlet-To-Neumann $\Lambda_{\sigma_\OO}$ è indipendente da $\OO(x)$ e vale
%\begin{Large}
\[ \Lambda_{\sigma_\OO} = \Lambda_\1 ,\]
%\end{Large}
cioè le due conducibilità sono assolutamente indistinguibili tramite sole misurazioni esterne.
\end{block}
\end{frame}


\begin{frame}{Conclusioni}
\pause
Supponiamo di costruire il nostro cloaking singolare nel guscio sferico $B_{R_2}\setminus B_{R_1}$ e che valga $R_1,R_2>>\lambda$, dove $\lambda$ è la lunghezza d'onda. Con questa ipotesi possiamo rappresentare il vettore di Poynting.
\pause
\begin{figure}[h] 
\centering 
\includegraphics[width=0.8\linewidth]{sweg.png}
\caption{Immagini ottenute da Pendry, Schurig e Smith tramite integrazione numerica delle equazioni di Maxwell}
\end{figure}
\end{frame}


\begin{frame}{Bibliografia}
%\bibliography{Bibliografia}
\bibliographystyle{abbrv}
%\nocite{*}

\begin{thebibliography}{99} 
\bibitem{HarryPotter} Kurt Brian, Tanya Leise, \textit{Impedance Imaging, Inverse Problems, and Harry Potter's Cloak}. (2009)

\bibitem{1} Kari Astala and Lassi Päivärinta, \textit{Calderón's Inverse Conductivity Problem in the Plane}. Annals of Mathematics, vol. 163, no. 1, 2006, pp. 265–299.

\bibitem{4} John M. Lee and Gunther Uhlmann, \textit{Determining anisotropic real‐analytic conductivities by boundary measurements}. (1989) Comm. Pure Appl. Math., 42: 1097-1112.

\bibitem{5} J. B. Pendry, D. Schurig, D. R. Smith, \textit{Controlling Electromagnetic Fields}

\bibitem{7} D. Schurig, J. J. Mock, B. J. Justice, S. A. Cummer, J. B. Pendry, A. F. Starr, D. R. Smith, \textit{Metamaterial Electromagnetic Cloak at Microwave Frequencies}. Science 10 Nov 2006 : 977-980

\end{thebibliography}
\end{frame}




\end{document}
